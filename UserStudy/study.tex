 % -*- root: ../medieninformatik-arbeit.tex -*-
\documentclass[../medieninformatik-arbeit.tex]{subfiles}
\begin{document}
\section{User Study}
\label{ch:study}
In order to proof the implemented functionality and interaction concepts in the web configurator where guiding users in the visualization and customization process in an enjoyable yet efficient way, a user study was developed. The user study was comprised of a thinking aloud usability test with a small group of participants and online test both including a questionnaire. For ensuring the configurator was importing the data from the Withings API the author wore the fitness tracker every day for a time span of 2 months. The specifics of the design of the user studies and the questionnaire as well as the obtained results will be discussed in the following sections. 

\subsection{Study Design}
The first part of the test for the usability of the web configurator was to perform a thinking aloud test. The thinking aloud protocol first introduced by Clayton Lewis in \cite{lewis1982using}, allows designers and developers to evaluate the usability of their system by letting a user speak out loud their thoughts as they use it. Users speak out of the content of their working memory which is driven by attention \cite{lewis1993task}. The basic concept of the thinking aloud protocol is to put users in front of a prototype of the system. Then developers have to chose a task for the user to solve in the prototype. Users are then filmed while they solve the task and speak their thoughts while using the prototype. Normally a thinking aloud usability test is performed with small number of participants raging from 5 to 10, although some studies have been made that show that a usability test with 20 participants reveals up to 95\% of the problems in a system \cite{vande2008beyond}. 

For the purposes of this work the thinking aloud test was designed as follows. The users would sit in front of a laptop running the web configurator from the web server, meaning not the local development version was ever used but the live version available to the public. An informative text was designed (see appendix \ref{App:userstudy:consent}) that explained users the purpose of the web configurator, the task they where to perform and the concept of thinking aloud. This text also included a consent form for allowing the author to record a screen cast of their performance with an audio recording of their voice. The task to solve by users was to create a new sculpture, manipulate it to their contempt and save the sculpture when finished. After the sculpture was saved users where redirected to the questionnaire done in the Google Froms platform \cite{googleforms}. Due that it was not required for the participants to have a their own Withings account, users started from the dashboard which gave them at the beginning an overview of the available data and also of the saved sculptures by other users. In this way it was possible to see if the designed process for creating a new sculpture was understandable for users. 

As discussed in the implementation chapter of this work, one of the requirements was to develop a working version and deploy it in a web server for the general public. So that anybody with a Withings account can log in and start visualizing their data. The purpose of this was to study how different data sets generate different sculptures. It was also of high interest for this work to receive feedback from the quantified self community, as the familiarity and interest for gaining insight from visualization of personal activity data is highly sought after in this community and activity sculptures are fairly new to the public. In order to achieve a high exposure of the web configurator in the quantified self community several platforms where chosen to present the web configurator. Although no exclusive forum for Withings users was found, a Withings user group was found in a sub page from the news aggregator website Reddit \cite{redditwithings}. The configurator was also introduced in the official quantified self forum \cite{qfselfforum}. Lastly the author contacted the organizers of the Munich quantified self meetup \cite{qfselfmeetup} for the possibility of introducing the web configurator in an upcoming meeting. Users that accessed the web configurator through the forum posts where first greeted with a consent dialog that explained the concepts of the project and stated it was only for research purposes related to this work. Only after users agreed with the terms of use they where able to login into the system. Also after saving a sculpture users where redirected to the questionnaire. 

To make the most out of online tester's participation, the use of an analytics tools was required. Analytics tools help web sites owners gather information about what pages, for how long and from where users visit. Through the gathered information website owners can make changes to improve the visitors experience and translate this to higher revenues \cite{peterson2004web}. Another functionality web analytics offer is the tracking of mouse clicks which are used to generate event logs and in some cases even heat-map images that visually display what parts of the interface receive the most clicks. The web configurator makes use of the Inspectlet web analytics tool \cite{inspectlet} to track traffic and generate heat-map images of click events. Another nicety of the Inspectlet service is the ability to record videos of user sessions. This provides almonst the same results as the screen-cast recordings performed for the thinking aloud test. Other analytics tools where considered for this work like Hotjar \cite{hotjar} and Crazyegg \cite{inspectlet} all of which provide more or less the same functionality but with different price ranges. Inspectlet showed to be easily implementable through a short code snippet that is embedded to the index.html file of the web configurator. 

\subsection{Questionnaire}
The questionnaire was designed to gather qualitative data about the implemented web configurator and the designed activity sculpture. In total 35 questions where developed to question users about 3 main subjects: web configurators in general, the activity sculpture web configurator and the activity sculpture. It was important to ask users about any previous experience with web configurators in order to better estimate the familiarity of users with such systems. Users where also asked about their perception about the value added to any product by making it customizable and if they thought customizable products tend to be selled at higher prices. The familiarity of users with fitness trackers and physical activity habits was important to be able to have an idea of how active participants were. Regarding the implemented activity sculpture configurator users were asked about the attractiveness of the interface, ease of use and learnability. Feedback for the designed controllers was also wanted, therefore the questionnaire included some questions about the complexity of the controls, the understandability of label descriptions and visual feedback from updatingthe sculpture. User's opinion regarding the activity sculpture is equally important as it is the the product of the configurator and will influence the user's experience with the configurator. In this area users were asked about which visualization style they found most appealing, any attachment they felt to the generated sculpture, interestingness, the likeness of them using the configurator in a commercial context, willingness to expand the interaction of the sculpture to wearable objects, complexity of the sculpture, interest in more sculpture variations  and the feasibility of them sharing the sculpture with friends and relatives.
Finally users were asked about the overall experience of using the activity sculpture configurator and were offered a field for writing any comments or thoughts about the configurator. 
The presented questionnaire can be found in the appendix \ref{App:userstudy:questionnaire}.


\subsection{Participants}
5 Participants for the thinking aloud
9 Survey responses



\subsection{Procedure}

For thinking aloud user study:
In the cafeteria, ask dudes if they would like to take part in it.
Sit down, make them read and sign the consent. Start screencast with audio recording and let them use the configurator

For remote user testing:
Send friends login credentials and let them use it


\subsection{Results}
Explain questionnaire results
Show heatmaps and explain a bit

45 visits from reddit
174 visits from qs forum

55 sessions recorded with Inspectlet

\subsection{Limitations}
Time constraints
Requiring Withings account/more brand support would be better 
Visitors of the web page had fear of the consent message. 
Inactivity of the meet up and specialized users
Web Server error because of oAuth callback

\end{document}
