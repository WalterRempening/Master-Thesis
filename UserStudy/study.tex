 % -*- root: ../medieninformatik-arbeit.tex -*-
\documentclass[../medieninformatik-arbeit.tex]{subfiles}
\begin{document}
\section{User Study}
\label{ch:study}
In order to proof the implemented functionality and interaction concepts in the web configurator for guiding users in the visualization and customization process in an enjoyable yet efficient way, a user study was developed. The user study was comprised of a thinking aloud usability test with a small group of participants and online test both including a questionnaire. For ensuring the configurator was importing the data from the Withings API the author wore the fitness tracker every day for a time span of 2 months. The specifics of the design of the user studies and the questionnaire as well as the obtained results will be discussed in the following sections. 

\subsection{Study Design}
The first part of the test for the usability of the web configurator was to perform a thinking aloud test. The thinking aloud protocol first introduced by Clayton Lewis in \cite{lewis1982using}, allows designers and developers to evaluate the usability of their system by letting a user speak out loud their thoughts as they use it. Users speak out of the content of their working memory which is driven by attention \cite{lewis1993task}. The basic concept of the thinking aloud protocol is to put users in front of a prototype of the system. Then developers have to chose a task for the user to solve in the prototype. Users are then filmed while they solve the task and speak their thoughts while using the prototype. Normally a thinking aloud usability test is performed with small number of participants raging from 5 to 10, although some studies have been made that show that a usability test with 20 participants reveals up to 95\% of the problems in a system \cite{vande2008beyond}. 

For the purposes of this work the thinking aloud test was designed as follows. The users would sit in front of a laptop running the web configurator from the web server, meaning not the local development version was ever used but the live version available to the public. An informative text was designed (see appendix \ref{App:userstudy:consent}) that explained users the purpose of the web configurator, the task they were to perform and the concept of thinking aloud. This text also included a consent form for allowing the author to record a screen cast of their performance with an audio recording of their voice. The task to solve by users was to create a new sculpture, manipulate it to their contempt and save the sculpture when finished. After the sculpture was saved users were redirected to the questionnaire done in the Google Froms platform \cite{googleforms}. Due that it was not required for the participants to have a their own Withings account, users started from the dashboard which gave them at the beginning an overview of the available data and also of the saved sculptures by other users. In this way it was possible to see if the designed process for creating a new sculpture was understandable for users. 

As discussed in the implementation chapter of this work, one of the requirements was to develop a working version and deploy it in a web server for the general public. So that anybody with a Withings account can log in and start visualizing their data. The purpose of this was to study how different data sets generate different sculptures. It was also of high interest for this work to receive feedback from the quantified self community, as the familiarity and interest for gaining insight from visualization of personal activity data is highly sought after in this community and activity sculptures are fairly new to the public. In order to achieve a high exposure of the web configurator in the quantified self community several platforms were chosen to present the web configurator. Although no exclusive forum for Withings users was found, a Withings user group was found in a sub page from the news aggregator website Reddit \cite{redditwithings}. The configurator was also introduced in the official quantified self forum \cite{qfselfforum}. Lastly the author contacted the organizers of the Munich quantified self meetup \cite{qfselfmeetup} for the possibility of introducing the web configurator in an upcoming meeting. Users that accessed the web configurator through the forum posts were first greeted with a consent dialog that explained the concepts of the project and stated it was only for research purposes related to this work. Only after users agreed with the terms of use they were able to login into the system. Also after saving a sculpture users were redirected to the questionnaire. 

To make the most out of online tester's participation, the use of an analytics tools was required. Analytics tools help web sites owners gather information about what pages, for how long and from where users visit. Through the gathered information website owners can make changes to improve the visitors experience and translate this to higher revenues \cite{peterson2004web}. Another functionality web analytics offer is the tracking of mouse clicks which are used to generate event logs and in some cases even heat-map images that visually display what parts of the interface receive the most clicks. The web configurator makes use of the Inspectlet web analytics tool \cite{inspectlet} to track traffic and generate heat-map images of click events. Another nicety of the Inspectlet service is the ability to record videos of user sessions. This provides almonst the same results as the screen-cast recordings performed for the thinking aloud test. Other analytics tools were considered for this work like Hotjar \cite{hotjar} and Crazyegg \cite{inspectlet} all of which provide more or less the same functionality but with different price ranges. Inspectlet showed to be easily implementable through a short code snippet that is embedded to the index.html file of the web configurator. 

\subsection{Questionnaire}
The questionnaire was designed to gather qualitative data about the implemented web configurator and the designed activity sculpture. In total 35 questions were developed to question users about 3 main subjects: web configurators in general, the activity sculpture web configurator and the activity sculpture. The questionnaire was composed of questions that users could answer in some cases by selecting given options or in other cases through Likert scales that went from strongly agree to strongly disagree. It was important to ask users about any previous experience with web configurators in order to better estimate the familiarity of users with such systems. Users were also asked about their perception about the value added to any product by making it customizable and if they thought customizable products tend to be sold at higher prices. The familiarity of users with fitness trackers and physical activity habits was important to be able to have an idea of how active participants were. Regarding the implemented activity sculpture configurator users were asked about the attractiveness of the interface, ease of use and learnability. Feedback for the designed controllers was also wanted, therefore the questionnaire included some questions about the complexity of the controls, the understandability of label descriptions and visual feedback from updatingthe sculpture. User's opinion regarding the activity sculpture is equally important as it is the the product of the configurator and will influence the user's experience with the configurator. In this area users were asked about which visualization style they found most appealing, any attachment they felt to the generated sculpture, interestingness, the likeness of them using the configurator in a commercial context, willingness to expand the interaction of the sculpture to wearable objects, complexity of the sculpture, interest in more sculpture variations  and the feasibility of them sharing the sculpture with friends and relatives.
Finally users were asked about the overall experience of using the activity sculpture configurator and were offered a field for writing any comments or thoughts about the configurator. 

The presented questionnaire can be found in the appendix \ref{App:userstudy:questionnaire}.

\subsection{Participants \& Procedure}
In a time span of 3 days participants were selected randomly by asking students in a university building to participate in the thinking aloud test, resulting in 5 participants gathered. For the thinking aloud test a table in a quiet area was set up with the laptop and the instructional text with the consent form. After reading and signing the consent, the recording of the screencast with audio was started  users started to use the configurator speaking their thoughts out loud. After the sculpture was saved the recording was stopped and users continued to answer the questionnaire. 

For the quantified self forum and the Withings Reddit subgroup one post was written in each platform and for a period of one month traffic from this website was tracked and screencast recorded with the Inspectelt service. 

\subsection{Results}
Between the thinking aloud and the online testers 9 survey answers were gathered. Participants had an average age of 25 ranging from 19 to 35. 6 of them were males and 3 females. The post about the web configurator in the Reddit Withings user group had 45 views  and in the Quantified self forum 174 views summing up a total of 219 viewers in the test month. The  Inspectlect analytics service recorded 55 sessions, because of the free version used for this work which was limited to 4 recordings per day. From these 55 recordings Inspectlet showed that the visitors came from distinct parts around the globe including China, France, the U.S., Mexico, Canada, Switzerland, United Kingdom and Denmark. Unfortunately the analytics show that the duration of the visits were as long as a couple of seconds. Causes of this user behavior can be the consent window that is displayed at the beginning. Maybe visitors get discouraged by the warnings regarding data privacy and because the website does not show them any more information that motivates them to stay users leave. Another problem encountered was an authentication bug in the application that after granting permission to the configurator to access their resources from the Withings server, redirected users to a ``Bad Gateway'' or ``500 Internal Server Error'' messages making it impossible for visitors to use the configurator. This was noted by a visitor from the Reddit post and it might have scared visitors off. Until the bug was fixed some weeks had already passed by and instead the author asked directly to friends to use the configurator with his own credentials. Added to this errors, it was not possible to present the configurator in the Munich Quantified self meetup due to the organizers not having enough speakers for the evening. Unfortunately the online testing phase was a failure and the goals of evaluating sculptures generated with different data and gathering feedback from the quantified self community were not achieved. 

From the participants that did answer the questionnaire the following results were gathered. Participants responded to have very low experience with web configurators or 3D software that also require them to navigate objects in 3D space with only one participant had used a web configurator before and two participants had intermediate to no experience with 3D software. The user that had experience with web configurators described his experience as neutral. Even though only one participant from the group had used a web configurator, everybody assured that customizable products seem more attractive to them and 66\% felt customizable products tend to be more expensive. Although web configurators seem to be not as widespread for this user sample, customization of product is perceived as an added value to the product and is attractive to customers. 
Regarding their activity habits almost half of participants does not have any exercise habits and one third exercise few times a week, leaving the rest to only sporadic work out session. The usage of tracking devices was only adopted by one user and stated to use fitness trackers under the other category. The same user confirmed he had experience a motivation boost from using it. Nonetheless, the rest of the users but one, had considered starting using a fitness tracker. Participants opinion regarding fitness trackers usage seems to be highly influenced by participant's activity habits as most participants are not very active, but nonetheless fitness trackers seem interesting to most users. 
Over half of the participants found the sculpture to be aesthetically appealing and 71.4\% found the normal visualization style to be the most attractive. The greater half of participants felt neutrally attached to the sculpture with the rest of participants leaning toward stronger attachment. Overall participants found it interesting to see their data visualized in a sculpture with 60\% answering positively. The survey showed participants would were unsure if they would use the configurator regularly to visualize their data as one half was leaning more towards using it and the other half against it. Participants showed to be neutrally interested in the possibility of paying to get their sculpture 3D printed but half of them were willing to wear their sculpture as an accessory. Sharing the sculpture with friends and relatives was perceived as positive and most users answered positively. Towards the need of a more accurate visualization of the data in the sculpture, users appeared to be rather neutral. Two thirds of participants were also rather satisfied with their sculpture and one third was neutral about it. 
The greater half of users expressed they found the configurator's user interface to be aesthetically appealing and also easy to use. The system was graded to be easy to learn by 77\% of participants. 70\% of users answered the labels describing the UI controls were helpful and the transformation the controls performed on the sculpture was also clear to 80\% of participants. Participants found the available configuration options to be enough and 90\% of participants described the sculpture updating instantly after changing controls to be very helpful. The dashboard and the sculpture gallery were regarded as helpful to get a summary of the gathered data and of the collected sculptures over time. Two thirds of participants answered they would have liked to have more sculpture designs to choose from.
Finally 80\% of users expressed they thought the configurator's functionality was useful and all of them said they would use it again. As final thought only one person expressed the variables showed in the activity chart to be somewhat confusing because activity and calories burned were combined in the same chart. The questionnaire results are also available in appendix \ref{App:survey:results}.

The thinking aloud usability test was performed successfully with only some minor technical difficulties with the Google Form which crashed once making it impossible for one participant to answer the questionnaire. The video material gathered from the thinking aloud tests was evaluated with the goal of finding errors in the user interface and also understand the thought process of users while utilizing the configurator. For this a spreadsheet was developed containing 4 fields for features missed by the user, UI widgets problems, user comments and remarks from the author. In general users performed fairly well and experimented with most features of the configurator. The widgets that were missed by users were the labels toggle missed by 4 users, the wireframe toggle missed by 2 users and only one user missed the geometry tab for the left control panel. Furthermore 2 users had difficulty operating the toggle button. It is important to mention that the toggle button implemented in the configurator with the Angular-material framework resembles the toggle button found in most smartphones nowadays which is primarily designed for touch gesture based interaction. Due that the configurator was designed as a desktop application the interaction with this kind of toggle buttons is not suitable for mouse operation. Users tried to wipe the button with the mouse when which did not activate the toggle. Instead users should have just clicked the toggle to make it change value. It was interesting to see how users were discovering the concepts behind the activity sculpture and the overall functionality of the configurator. The positioning of the new sculpture button in the dashboard was easy for users to find and was described as an``obvious'' way to create the new sculpture. Two users commented it was difficult for them understanding what the interpolated mode did to the sculpture. The purpose of the sculpture was questioned by two users, one of them stating that data visualization should be precise. The same user was expressed to be irritated by the normalization of the data and described it as illogical. Users also made comments when they discovered that the data in the sculpture is visualized overt time and axis can be added to the sculpture. One user was surprised to see how much data is possible to gather about ones activity and also commented it was difficult to note any difference in the sculpture when adding many variables. On the author's remarks, users sooner or later intuitively navigated the sculpture by clicking and dragging even though there are no instructions in the configurator. In general users did not make any comments about the dashboard an started the process of creating a new sculpture right away. The ``SPO2'' label was also new to most users as many of them asked its meaning. Interestingly one user did not understand that he was able to add variables to the sculpture by clicking on the buttons and asked where could he enter new data. An interesting observation of self reflection was made by a user while navigating the sculpture stating ``this day I made a lot of intense activity'' showing how even though it was not their data being visualized participants tried to see the sculpture as their own. 

Because for the online test and for the thinking aloud test the deployed configurator in the public server was used all participant's click events were also tracked by the Inspectlet analytics. Inspectlet generates three different heatmap images: click events, eye-tracking and scroll heatmaps. The eye-tracking heatmap is not generated through the use of eye cameras but is an approximation based on mouse movements and clicks. Scroll heatmaps allow web site owners to see until which point users scroll through the page. For the purposes of this work only the click heatmap was used (see figure \ref{fig:heatmap} and appendix \ref{App:survey:heatmap:click}). The generated heatmap shows through the bigger marks how users clicked mostly on the left control panel. Unfortunately its not possible to see if users were selecting variables or operating controls in the top area. Nonetheless the heatmap shows users experimented with different variables and controls heavily. In the middle section many marks are also visible from navigating the model in 3D space, enforcing the intuitiveness users have to manipulate the sculpture in 3D. Because of the bigger mark on top of the data tab it seems that users experimented more with adding variables than with geometry parameters. On the right side it seems users did not experiment a lot with the color picker as it remained fairly clean of marks. The export tab seems to have been selected frequently as it presents heat marks.

\begin{figure}[h]
\captionsetup{width=\textwidth}
\begin{center}
  \includegraphics[width=0.7\textwidth]{Appendix/questionnaire/click_heatmap}
  \caption{Inspelect analytics generated click heatmap}
\label{fig:heatmap}
\end{center}
\end{figure}

To summarize the results, it seems web configurators are not as common as thought, but users are willing to experiment and the customization of products remains interesting for most users. The web configurator performed very well as users found it to be ease to operate and also to learn. Other than the toggle button all controls performed well and the labels were also understood by users. Interestingly users spend more time experimenting with different combinations of variables mapped to the sculpture than customizing the geometry. The activity sculpture was also perceived as aesthetically appealing and users seem to have a neutral opinion towards the level of abstraction used in the sculpture. The interpolated visual style seems to be somewhat confusing to users as they did not understood well what was happening in the sculpture. It seems the usefulness of activity sculptures is not entirely understood as it was questioned by some users. Even though the online test failed on the forums and with the meetup, the usability test and the few online testes still produced helpful insight. 

\subsection{Limitations}
Among the limitations of the user study was the short available time that was left due to the prolonged development phase. Initially the goal was to perform several smaller user studies but again the development required more time than expected and the user studies had to be shortened to a usability test and an online test which unfortunately failed. 

Limiting the configurator to work only with the Withings API had its advantages and disadvantages. It provided users a very simple form of importing their data but Withings the user base seems to not have a dedicated forum and the sense of community for sharing experiences between users is not present. As in the research phase for the best fitness tracker to work with these aspects were not entirely considered the user study suffered. A solution for this would have been to contact for at least a small group of Withings or other brand users and perform a more in depth study with them. By attending the Munich Quantified self meetup the author expected to reach more Withings users and get more feedback from more advanced quantified self practitioners. Sadly the inactivity of the community in Munich did not allow for a meeting in the time frame of the development of this work. 

The login bug that impeded users to access the web page affected the success of the online study greatly. The difficult part of fixing the bug was its reproducibility. The Withings authentication requires the URL address starting the \textit{OAuth} flow had to be the same as in the callback. The posted URL in the forums contained ``www.'' at the beginning and in the callback the URL did not have the ``wwww.'' at the beginning which translated in the authentication issue. Unfortunately the bug took one week to fix in between performing the user study. Another factor for the failure of the online test seems to be the consent message displayed at the beginning. Admittedly not many options were considered for displaying the consent message. In this place it would have been helpful to give visitors a friendlier welcome site, showcasing some attractive aspects of the project inviting users to log in and try out the configurator. 

The conclusion of this work and the possible improvements to the configurator will be discussed in the following chapters respectively. 

\end{document}
