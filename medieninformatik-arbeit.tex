\documentclass[11pt,a4paper,twoside]{article}
% \synctex=1

% LaTeX-Umsetzung der "Richtlinien f�r Projekt- und Diplomarbeiten"
% der LFE Medieninformatik, LMU M�nchen. (Autor: Richard Atterer, 27.9.2006, 23.10.2007), Bug-Fixing Mark Kaczkowski (23.6.2008)

\usepackage[T1]{fontenc} % sonst geht \hyphenation nicht mit Umlauten
\usepackage[latin1]{inputenc} % man kann schreiben ����, statt "a"o"u"s
%\usepackage[utf8]{inputenc} % wie oben, aber UTF-8 als Encoding statt ISO-8859-1 (latin1)
\usepackage[ngerman,english]{babel} % deutsche Trennregeln, "Inhaltsverzeichnis" etc.
%\usepackage{ngerman} % Alternative zum Babel-Paket oben
\usepackage{mathptmx} % Times-Roman-Schrift (auch f�r mathematische Formeln)
\usepackage{subfiles}
\usepackage{caption}
\usepackage{upquote}
\usepackage{pdfpages}
\usepackage[official]{eurosym}
\usepackage{listings} % For displaying js source code

%\usepackage{url} % Cross references
% Zum Setzen von URLs
\usepackage{color}
\definecolor{darkred}{rgb}{.25,0,0}
\definecolor{darkgreen}{rgb}{0,0.2,0}
\definecolor{darkmagenta}{rgb}{.2,0,.2}
\definecolor{darkcyan}{rgb}{0,.15,.15}
\usepackage[plainpages=false,bookmarks=true,bookmarksopen=true,colorlinks=true,linkcolor=darkred,citecolor=darkgreen,filecolor=darkmagenta,menucolor=darkred,urlcolor=darkcyan]{hyperref}

% pdflatex: Bilder in den Formaten .jpeg, .png und .pdf
% latex: Bilder im .eps-Format
\usepackage{graphicx}

\usepackage{fancyhdr} % Positionierung der Seitenzahlen
\fancyhead[LE,RO,LO,RE]{}
\fancyfoot[CE,CO,RE,LO]{}
\fancyfoot[LE,RO]{\Roman{page}}
\renewcommand{\headrulewidth}{0pt}
\setlength{\headheight}{13.6pt} % behebt headheight Warning


% Korrektes Format f�r Nummerierung von Abbildungen (figure) und
% Tabellen (table): <Kapitelnummer>.<Abbildungsnummer>
\makeatletter
\@addtoreset{figure}{section}
\renewcommand{\thefigure}{\thesection.\arabic{figure}}
\@addtoreset{table}{section}
\renewcommand{\thetable}{\thesection.\arabic{table}}



% JS, HTML and CSS listings
\definecolor{lightgray}{rgb}{0.95, 0.95, 0.95}
\definecolor{darkgray}{rgb}{0.4, 0.4, 0.4}
%\definecolor{purple}{rgb}{0.65, 0.12, 0.82}
\definecolor{editorGray}{rgb}{0.95, 0.95, 0.95}
\definecolor{editorOcher}{rgb}{1, 0.5, 0} % #FF7F00 -> rgb(239, 169, 0)
\definecolor{editorGreen}{rgb}{0, 0.5, 0} % #007C00 -> rgb(0, 124, 0)
\definecolor{orange}{rgb}{1,0.45,0.13}      
\definecolor{olive}{rgb}{0.17,0.59,0.20}
\definecolor{brown}{rgb}{0.69,0.31,0.31}
\definecolor{purple}{rgb}{0.38,0.18,0.81}
\definecolor{lightblue}{rgb}{0.1,0.57,0.7}
\definecolor{lightred}{rgb}{1,0.4,0.5}

% CSS
\lstdefinelanguage{CSS}{
  keywords={color,background-image:,margin,padding,font,weight,display,position,top,left,right,bottom,list,style,border,size,white,space,min,width, transition:, transform:, transition-property, transition-duration, transition-timing-function}, 
  sensitive=true,
  morecomment=[l]{//},
  morecomment=[s]{/*}{*/},
  morestring=[b]',
  morestring=[b]",
  alsoletter={:},
  alsodigit={-}
}

% JavaScript
\lstdefinelanguage{JavaScript}{
  morekeywords={typeof, new, true, false, catch, function, return, null, catch, switch, var, if, in, while, do, else, case, break},
  morecomment=[s]{/*}{*/},
  morecomment=[l]//,
  morestring=[b]",
  morestring=[b]'
}

\lstdefinelanguage{HTML5}{
  language=html,
  sensitive=true,   
  alsoletter={<>=-},    
  morecomment=[s]{<!-}{-->},
  tag=[s],
  otherkeywords={
  % General
  >,
  % Standard tags
    <!DOCTYPE,
  </html, <html, <head, <title, </title, <style, </style, <link, </head, <meta, />,
    % body
    </body, <body,
    % Divs
    </div, <div, </div>, 
    % Paragraphs
    </p, <p, </p>,
    % scripts
    </script, <script,
  % More tags...
  <canvas, /canvas>, <svg, <rect, <animateTransform, </rect>, </svg>, <video, <source, <iframe, </iframe>, </video>, <image, </image>, <header, </header, <article, </article
  },
  ndkeywords={
  % General
  =,
  % HTML attributes
  charset=, src=, id=, width=, height=, style=, type=, rel=, href=,
  % SVG attributes
  fill=, attributeName=, begin=, dur=, from=, to=, poster=, controls=, x=, y=, repeatCount=, xlink:href=,
  % properties
  margin:, padding:, background-image:, border:, top:, left:, position:, width:, height:, margin-top:, margin-bottom:, font-size:, line-height:,
    % CSS3 properties
  transform:, -moz-transform:, -webkit-transform:,
  animation:, -webkit-animation:,
  transition:,  transition-duration:, transition-property:, transition-timing-function:,
  }
}

\lstdefinestyle{htmlcssjs} {%
  % General design
%  backgroundcolor=\color{editorGray},
  basicstyle={\footnotesize\ttfamily},   
  frame=tb,
  % line-numbers
  xleftmargin={0.75cm},
  numbers=left,
  stepnumber=1,
  firstnumber=1,
  numberfirstline=true, 
  % Code design
  identifierstyle=\color{black},
  keywordstyle=\color{blue}\bfseries,
  ndkeywordstyle=\color{editorGreen}\bfseries,
  stringstyle=\color{editorOcher}\ttfamily,
  commentstyle=\color{darkgray}\ttfamily,
  % Code
  language=HTML5,
  alsolanguage=JavaScript,
  alsodigit={.:;},  
  tabsize=2,
  showtabs=false,
  showspaces=false,
  showstringspaces=false,
  extendedchars=true,
  breaklines=true,
  % German umlauts
  literate=%
  {�}{{\"O}}1
  {�}{{\"A}}1
  {�}{{\"U}}1
  {�}{{\ss}}1
  {�}{{\"u}}1
  {�}{{\"a}}1
  {�}{{\"o}}1
}
\makeatother

\sloppy % Damit LaTeX nicht so viel �ber "overfull hbox" u.�. meckert

% R�nder
\addtolength{\topmargin}{-16mm}
\setlength{\oddsidemargin}{25mm}
\setlength{\evensidemargin}{35mm}
\addtolength{\oddsidemargin}{-1in}
\addtolength{\evensidemargin}{-1in}
\setlength{\textwidth}{15cm}
\addtolength{\textheight}{34mm}

%______________________________________________________________________

\begin{document}

\pagestyle{empty} % Vorerst keine Seitenzahlen
\pagenumbering{alph} % Unsichtbare alphabetische Nummerierung

\makeatletter
\@addtoreset{lstlisting}{section}
\renewcommand\thelstlisting{\thesection.\arabic{lstlisting}}
\makeatother

\begin{center}
\textsc{University of Munich}\\
Department ``Institute for Informatics''\\
Education and Research Units Media Informatics\\
Prof.\ Dr.\ Heinrich Hu�mann


\vspace{5cm}
{\large\textbf{Master Thesis}}\vspace{.5cm}

{\LARGE Web-Based Creator for Activity Sculptures}\vspace{1cm}

{\large Walter Rempening D�az}\\\href{mailto:me@walterrempening.com}{me@walterrempening.com}

\end{center}
\vfill

\begin{tabular}{ll}
Working Time: & 1. 12. 2014 to 8. 6. 2015\\
Supervisor: & Simon Stusak\\
Responsible Professor: & Prof. Dr. Andreas Butz
\end{tabular}
%______________________________________________________________________

\cleardoublepage
\section*{Acknowledgements}

\cleardoublepage

\selectlanguage{english}
\section*{Abstract}
The recollection of personal activity data has been greatly facilitated by the
increasing amount of applications and devices that encourage users to measure
their activity with the primary goal of health improvement. These
devices range from mobile applications taking advantage of smartphone sensors to
dedicated fitness trackers presented as modern watches and bracelets. Apart from the
analytical insights about the data obtained through classic data
visualizations, it is also possible to visualize the information through
physical objects also known as activity sculptures. It has been shown that
activity sculptures have a positive influence in users making them feel rewarded
for their active lifestyle. To further study the process of visualizing activity
information into sculptures a web-based activity sculpture creator was
developed. This tool takes advantage of modern web
technologies and offers a platform in which users can import their data and
allows them to experiment creating variations of an activity sculpture which can also
be exported for 3D printing. For the development of the configurator current
product customization platforms were analyzed for gathering best practices in
user interface and interaction design. In order for users to have a sculpture
with a high degree of data mapping variability, 4 
different sculpture prototypes were developed. For the validation of the
configurator an online version was released and a user study was performed.
User feedback showed that our prototype was easy to operate and that the
obtained sculptures were appealing and meaningful to them.

\selectlanguage{ngerman}
\section*{Zusammenfassung}
Die Sammlung pers�nlicher Aktivit�tsdaten wurde durch die zahlreiche Anzahl an
Anwendungen und Ger�te enorm vereinfacht. Diese Anwendungen und
Ger�tschaften, die haupts�chlich das Ziel haben, Nutzer zu einem aktiven
Lebensstil zu ermutigen, k�nnen in Smartphones, wo sie die Vielfalt an Sensoren
ausnutzen oder als tragbare Accessoires wie moderne Uhren oder Armb�nder gefunden
werden. Abgesehen davon, dass klassische Datenvisualisierungen Einblicke in den
Aktivit�tsdaten verschaffen k�nnen, ist es auch m�glich den Datensatz durch
physikalische Objekte, auch als Aktivit�tsskulpturen bekannt, zu visualisieren.
Es wurde bewiesen, dass Aktivit�tsskulpturen Nutzer positiv beeinflussen, da die
Nutzer sich f�r ihren aktiven Lebensstil belohnt f�hlen. Um den Prozess der
Visualisierung von Information in Skulpturen weiter zu forschen wurde ein
Web-Konfigurator f�r Aktivit�tsskulpturen entwickelt. Durch die Nutzung moderner
Web-Technologien erh�lt der Nutzer eine Platform die ihm es erlaubt seine Daten
unkompliziert zu imporiteren und erm�glicht ihn die Gestaltung einer 3D
druckbaren Skulptur. F�r die Entwicklung des Konfigurators, wurden aktuelle
Konfiguratoren analysiert mit dem Ziel Best-Practices im Bereich des Interface-
und Interaktionsdesigns zu sammeln. Um den Nutzer eine breite Vielfalt an
m�glichen Anpassungen f�r die Skulptur, wurden 4 verschiedene
Skulptur-Prototypen entwickelt. Letztendlich wurden f�r die Validierung des
Prototyps eine online Demoversion ver�ffentlicht und eine Nutzerstudie
durchgef�hrt. Die Resonanz der Nutzer zeigte, dass unser Prototyp einfach zu
bedienen ist und, dass die entstandene Skulptur ansprechend wund sinnvoll war. 


\selectlanguage{english}

\cleardoublepage
\section*{Task Definition}
Activity Sculptures are physical (3D printed) representations of personal
tracking data (e.g.\ step count) that dwell between the artistic and the abstract. 
For this master's thesis the student will develop a web configurator that will
allow to individually create said activity sculptures (a similar example can be
seen in
\href{https://www.shapeways.com/creator/statement_vase}{www.shapeways.com/creator/statement\_vase}).

The focus of the thesis will be the development of interaction concepts and
their implementation in the configurator. The concepts will be examined and
improved in smaller iterative user studies. Another important aspect is a
seamless and easy import of external tracking data (e.g.\ export data from tracking apps). The result should be a stable working prototype that can be used
for follow-up works.

\paragraph{Possible research questions}
\begin{itemize}
\item What interaction concepts are possible? What are their advantages and
disadvantages?
\item What degree of freedom is possible and meaningful while designing a visualization?
\item What is a possible design space for said activity sculptures?
\end{itemize}
\paragraph{Tasks}
\begin{itemize}
\item Research and related works (e.g. data visualization, configurators)
\item Development of interaction concepts
\item Concept implementation
\item Planing and executing several small user studies
\item Written thesis and presentation of work
\end{itemize}
\paragraph{Requirements}
\begin{itemize}
\item Programming skills in web development and computer graphics
\end{itemize}
\vfill % Sorgt daf�r, dass das Folgende an das Seitenende rutscht

\noindent I confirm that I independently prepared the thesis and that I used only
the references and auxiliary means indicated in the thesis. 
\bigskip\noindent 
\\Munich, \today

\vspace{4ex}\noindent\makebox[7cm]{\dotfill}

%______________________________________________________________________

\cleardoublepage
\pagestyle{fancy}
\pagenumbering{roman} % R�mische Seitenzahlen
\setcounter{page}{1}
% Inhaltsverzeichnis erzeugen
\tableofcontents

%______________________________________________________________________

\cleardoublepage
\pagestyle{empty}
\null\newpage
\null\newpage
% Arabische Seitenzahlen
\pagestyle{fancy}
\pagenumbering{arabic}
\setcounter{page}{1}
% Ge�ndertes Format f�r Seitenr�nder, arabische Seitenzahlen
\fancyhead[LE,RO]{\rightmark}
\fancyhead[LO,RE]{\leftmark}
\fancyfoot[LE,RO]{\thepage}

%_____________________________________________________________________
% subfile Kapitel
%______________________________________________________________________
\subfile{Introduction/intro}
%\_____________________________________________________________________

\clearpage
\subfile{RelatedWork/content}
%______________________________________________________________________

\clearpage
\subfile{Prototype/proto}
%______________________________________________________________________

\clearpage
\subfile{Configurator/conf}
%______________________________________________________________________

\clearpage
\subfile{UserStudy/study}
%______________________________________________________________________

\clearpage
\subfile{Conclusion/content}
%______________________________________________________________________

\clearpage
\subfile{FutureWork/content}
%______________________________________________________________________
%Abbildungsverzeichnis erzeugen - normalerweise nicht n�tig
\clearpage
\listoffigures
%______________________________________________________________________
\clearpage
\lstlistoflistings
%______________________________________________________________________
\cleardoublepage
\bibliographystyle{abbrv}
\bibliography{ma-thesis}

%______________________________________________________________________
\cleardoublepage
\subfile{Appendix/content}

%________________________________________________________________________________

\fancyhead[LE,RO,LO,RE]{} % Keine Kopfzeile mehr oben auf jeder Seite
\cleardoublepage
\addcontentsline{toc}{section}{Contents of the enclosed CD}
\section*{Contents of the enclosed CD}
\paragraph{Thesis}
\begin{itemize}
\item \LaTeX\ Document   
\item PDF File 
\end{itemize}

\paragraph{Activity Sculpture Web Configurator}
\begin{itemize}
\item Prototype sketches  
\item Source code
\item Instructions for deployment  
\item Login Data

\end{itemize}    

\paragraph{Sculptures}
\begin{itemize}
\item Prototype sketches 
\item.stl 3D print ready example files 
\end{itemize}

\paragraph{User Study}
\begin{itemize}
\item Questionnaire 
\item Results
\item Heat map images
\end{itemize}      

\end{document}