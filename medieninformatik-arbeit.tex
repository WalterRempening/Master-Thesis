\documentclass[11pt,a4paper,twoside]{article}
\synctex=1
% LaTeX-Umsetzung der "Richtlinien f�r Projekt- und Diplomarbeiten"
% der LFE Medieninformatik, LMU M�nchen. (Autor: Richard Atterer, 27.9.2006, 23.10.2007), Bug-Fixing Mark Kaczkowski (23.6.2008)

\usepackage[T1]{fontenc} % sonst geht \hyphenation nicht mit Umlauten
\usepackage[latin1]{inputenc} % man kann schreiben ����, statt "a"o"u"s
%\usepackage[utf8]{inputenc} % wie oben, aber UTF-8 als Encoding statt ISO-8859-1 (latin1)
\usepackage[ngerman,english]{babel} % deutsche Trennregeln, "Inhaltsverzeichnis" etc.
%\usepackage{ngerman} % Alternative zum Babel-Paket oben
\usepackage{mathptmx} % Times-Roman-Schrift (auch f�r mathematische Formeln)
%\usepackage{natbib} % Cross references
% Zum Setzen von URLs
\usepackage{color}
\definecolor{darkred}{rgb}{.25,0,0}
\definecolor{darkgreen}{rgb}{0,0.2,0}
\definecolor{darkmagenta}{rgb}{.2,0,.2}
\definecolor{darkcyan}{rgb}{0,.15,.15}
\usepackage[plainpages=false,bookmarks=true,bookmarksopen=true,colorlinks=true,
  linkcolor=darkred,citecolor=darkgreen,filecolor=darkmagenta,
  menucolor=darkred,urlcolor=darkcyan]{hyperref}

% pdflatex: Bilder in den Formaten .jpeg, .png und .pdf
% latex: Bilder im .eps-Format
\usepackage{graphicx}

\usepackage{fancyhdr} % Positionierung der Seitenzahlen
\fancyhead[LE,RO,LO,RE]{}
\fancyfoot[CE,CO,RE,LO]{}
\fancyfoot[LE,RO]{\Roman{page}}
\renewcommand{\headrulewidth}{0pt}
\setlength{\headheight}{13.6pt} % behebt headheight Warning

% Korrektes Format f�r Nummerierung von Abbildungen (figure) und
% Tabellen (table): <Kapitelnummer>.<Abbildungsnummer>
\makeatletter
\@addtoreset{figure}{section}
\renewcommand{\thefigure}{\thesection.\arabic{figure}}
\@addtoreset{table}{section}
\renewcommand{\thetable}{\thesection.\arabic{table}}
\makeatother

\sloppy % Damit LaTeX nicht so viel �ber "overfull hbox" u.�. meckert

% R�nder
\addtolength{\topmargin}{-16mm}
\setlength{\oddsidemargin}{25mm}
\setlength{\evensidemargin}{35mm}
\addtolength{\oddsidemargin}{-1in}
\addtolength{\evensidemargin}{-1in}
\setlength{\textwidth}{15cm}
\addtolength{\textheight}{34mm}
%______________________________________________________________________

\begin{document}

\pagestyle{empty} % Vorerst keine Seitenzahlen
\pagenumbering{alph} % Unsichtbare alphabetische Nummerierung

\begin{center}
\textsc{Ludwig-Maximilians-Universit�t M�nchen}\\
Department ``Institut f�r Informatik''\\
Lehr- und Forschungseinheit Medieninformatik\\
Prof.\ Dr.\ Heinrich Hu�mann


\vspace{5cm}
{\large\textbf{Projektarbeit}}\vspace{.5cm}

{\LARGE \LaTeX-Vorlage f�r Projekt- und Diplomarbeiten}\vspace{1cm}

{\large Walter Rempening
  Diaz}\\\href{mailto:name@example.org}{w.rempening@campus.lmu.de}

\end{center}
\vfill

\begin{tabular}{ll}
Bearbeitungszeitraum: & 1. 12. 2014 bis 1. 6. 2015\\
Betreuer: & Ludwig Lehrstuhlmitarbeiter\\
%Externer Betreuer: & Manfred Manager\\
Verantw. Hochschullehrer: & Prof. Butz ODER Prof. Hu�mann
\end{tabular}
%______________________________________________________________________

\clearpage
\section*{Zusammenfassung}

Kurzzusammenfassung der Arbeit, maximal 250 W�rter.

\selectlanguage{english}
\section*{Abstract}

Short abstract of the work, maximum of 250 words.

\selectlanguage{ngerman}
\clearpage
\section*{Aufgabenstellung}

Kopie der Original-Aufgabenstellung

\vfill % Sorgt daf�r, dass das Folgende an das Seitenende rutscht

\noindent Ich erkl�re hiermit, dass ich die vorliegende Arbeit
selbstst�ndig angefertigt, alle Zitate als solche kenntlich gemacht
sowie alle benutzten Quellen und Hilfsmittel angegeben habe.

\bigskip\noindent M�nchen, \today

\vspace{4ex}\noindent\makebox[7cm]{\dotfill}

%______________________________________________________________________

\cleardoublepage
\pagestyle{fancy}
\pagenumbering{roman} % R�mische Seitenzahlen
\setcounter{page}{1}

% Inhaltsverzeichnis erzeugen
\tableofcontents

%Abbildungsverzeichnis erzeugen - normalerweise nicht n�tig
%\cleardoublepage
%\listoffigures
%______________________________________________________________________

\cleardoublepage

% Arabische Seitenzahlen
\pagenumbering{arabic}
\setcounter{page}{1}
% Ge�ndertes Format f�r Seitenr�nder, arabische Seitenzahlen
\fancyhead[LE,RO]{\rightmark}
\fancyhead[LO,RE]{\leftmark}
\fancyfoot[LE,RO]{\thepage}

\section{Einleitung}

%______________________________________________________________________

% Der Befehl \cleardoublepage erscheint nur vor \section, nicht vor
% den "kleineren" Gliederungsbefehlen wie \subsection!
\cleardoublepage % Neue rechte Seite anfangen
\section{Hauptteil}

\begin{figure}%[btph]
  %% Datei ``beispielbild.eps'' oder ``beispielbild.png'', zentriert
  %\begin{center}\includegraphics{beispielbild}\end{center}

  %% Datei auf 8cm Breite verkleinert/vergr��ert
  %\includegraphics[width=8cm]{beispielbild}
  %% Datei auf ganze Breite des Texts vergr��ert
  %\includegraphics[width=\columnwidth]{beispielbild}
  %% Datei auf 60% der Textbreite verkleinert/vergr��ert
  %\includegraphics[width=.6\columnwidth]{beispielbild}
  %% Weitere Optionen (Ausschnitt, drehen etc.) in der Doku zum graphicx-Paket

  \begin{center}\LARGE [BILD]\end{center}
  \caption{Bildunterschrift}
  \label{fig:beispielbild}
\end{figure}

\chapter{Introduction}
\label{ch:intro}

\section{Introduction}
\subsection{Motivation}
\subsection{Problem definition}
\subsection{Goals}
\subsection{Content overview}


Siehe was cooles~\ref{fig:beispielbild} oder einschl�gige Literatur, z.B.
\cite{ahu61} oder \cite{ab94} oh ja ziemlich geil das ding
\bigskip % Gr��erer Abstand zum vorherigen Absatz


\subsection{Medien}

\begin{figure}
  \begin{center}\LARGE [BILD]\end{center}
  \caption{Noch ein Bild}
  \label{fig:beispielbild2}
\end{figure}

\begin{itemize}
  \item Gesellschaftliche Medien
  \item Technische Medien
\end{itemize}


\subsection{Informatik}


\subsection{Medieninformatik}

\begin{description}
  \item[Medienwirkung:] Ein Spezialfach der Kommunikationswissenschaft. F�r das erfolgreiche Studium des Anwendungsfachs Mediengestaltung ist eine k�nstlerische Begabung erforderlich.
  \item[Medienwirtschaft:] Ein Spezialfach der Betriebswirtschaftslehre
  \item[Mediengestaltung:] Ein Spezialfach der Kunstwissenschaft
\end{description}

\subsubsection{Was Sie schon immer wissen wollten, aber nie zu fragen
  wagten}

\paragraph{�berschrift}
Diese �berschrift erscheint fettgedruckt am Anfang des Absatzes.

\subsubsection{Was Sie nicht wissen wollten}

Text text textextext\footnote{Oder so �hnlich}.

%\_____________________________________________________________________

\cleardoublepage
\section{Zusammenfassung}

\begin{figure}
  \begin{center}\LARGE [BILD]\end{center}
  \caption{Bild}
  \label{fig:beispielbild3}
\end{figure}
%______________________________________________________________________

\cleardoublepage
\fancyhead[LE,RO,LO,RE]{} % Keine Kopfzeile mehr oben auf jeder Seite
\section*{Inhalt der beigelegten CD}
%______________________________________________________________________

\cleardoublepage

\bibliographystyle{abbrv}
\bibliography{ma-thesis}

\end{document}
