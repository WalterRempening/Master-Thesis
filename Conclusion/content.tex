 % -*- root: ../medieninformatik-arbeit.tex -*-
\documentclass[../medieninformatik-arbeit.tex]{subfiles}
\begin{document}
\section{Conclusion}
\label{ch:conclusion}
As a starting point this work introduced the reader to product customization systems starting with a summary of the mass customization movement and the impact new manufacturing technologies have had in the markets ability to offer unique products. By showing current examples of web configurators in a commercial context essential parts of these systems were discussed providing the reader a summary of what design approaches and implementations have proved to offer customers outstanding customizable products. Next the reader was introduced to the topic of activity sculptures a data visualization topic that has been able to be explored also through advancements in additive manufacturing which allow designers and engineers to translate visualizations from screen to physical space. The main properties and benefits of activity sculptures were introduced by showcasing current work in research and the industry that made use of web configurators to prove the design and manufacturing of activity sculptures can be done nowadays. A short review of available fitness tracker devices was offered with the goal of showcasing the different kinds of trackers available in the market and choosing one for the development of the activity sculpture web configurator. 

This work provided readers with insight in the design process of both the web configurator and the activity sculptures. For this work four different activity sculptures were developed: the 3D graph, the activity landscape, the activity flora and the activity vase. Each of the sculptures had unique design properties and were inspired from interesting concepts in order to provide a wide range of customization options to the user. The final decision was made based on a developed taxonomy based on Vande Moere's work \cite{vande2009analyzing}. The activity vase was chosen as it allowed for more data mappings and the abstract design provided many possibilities for expanding to other interaction forms. The configurator was designed in an iterative approach where three different prototypes were developed. Each prototype guided the user through the customization process with different degrees of complexity. For the final decision simplicity and clarity were important factors that led to chose the second prototype for the implementation. 

The chosen designs were then implemented using cutting edge technologies such as the MEAN stack, a powerful set of tools based on the JavaScript programming language that enable developers to write the backend and frontend parts of the application completely in one language. The architecture of the configurator was presented and provided code examples to better explain the utilized development concepts. The configurator made also use of WebGL for visualizing the sculpture in 3D within the browser providing users with an almost desktop experience. Modern web technologies are perfectly suitable for developing highly interactive visualization applications that interact with other web services for data extraction. 

The implemented configurator was then tested through the thinking aloud usability test with the help of 5 participants who provided helpful insight about which UI controls were hard to operate and showed navigation in 3D space was performed intuitively. Interestingly enough, users experimented a lot with mapping different variables to the sculpture, comparing how variables related to each other in the sculpture. Unfortunately the designed online test failed due to an error in the login section and because of the skepticism of the communities where the configurator was presented. A questionnaire answered by 9 participants showed the implemented interaction concepts in the web configurator were understandable and easy to learn. Users expressed the provided functionality was enough for them but they would like to see more different sculpture designs. The activity sculpture was aesthetically appealing to users and they were even ready to share their designed sculptures with friends or even wear them. 

To conclude this thesis, the web-based activity sculpture creator is an example of what available web technologies are capable of achieving. Even though activity sculptures and web configurators are from different fields, this work proved again how by taking knowledge from web configurator design, a tool that provides guidance to users in the design of activity sculptures can be developed. Through this work it was demonstrated that users can be more intensively included in the design process of activity sculptures offering enjoyable user experiences and simplified workflows.

\end{document}