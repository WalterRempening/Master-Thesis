 % -*- root: ../medieninformatik-arbeit.tex -*-
\documentclass[../medieninformatik-arbeit.tex]{subfiles}
\begin{document}
\section{Implementation}
\label{ch:configurator}
The implementation of the activity sculpture configurator and the activity sculpture were undertaken in a period of around two months. The developed web configurator in this work makes use of cutting edge technologies for both web development and 3D visualization. This project is a demonstration of the capabilities of current web technologies that allow the developing of fully functional data visualization systems that offer integration with other web services for data interchange and support real time interactive 3D visualization.

This chapter will discuss all aspects of the technical implementation of the prototypes discussed in chapter \ref{ch:proto}. The technical requirements for the system will be discussed in the first section followed by an overview of the used technologies. Further on the modules conforming the system and their relationship will be explained in the architecture section. THe configurator section covers the implementation of the controls used by the user to manipulate the sculpture and the generation of the sculpture which are the main functionalities covered in the front end side of the application. The integration of the Whithings API and the manipulation performed on the received data are covered in the backend section of the chapter. Finally an overview of the challenges encountered while developing the system is presented. 

\subsection{Requirements}
Through the analysis of current works described in chapter \ref{ch:related},the requirements of the prototypes in chapter \ref{ch:proto} and in view of the forthcoming user study the functionality and technical requirements of the web-based activity sculpture creator were formulated as follows.

\begin{itemize}
	\item Implement the designed configurator workflow views
	\item Users have to be able to login with their personal Withings account
	\item Query user's data with the Withings API
	\item Store user's data and profile information in a data base
	\item Implement user interface control widgets like buttons, sliders and toggles for the customization area of the configurator
	\item Implement the designed activity sculpture geometry
	\item The visualization of the sculpture has to be rendered in real time  
	\item Changes made in the configurator have to be reflected in the sculpture instantly. avoid the use of update buttons 
	\item Users have to be able to navigate the sculpture in 3D space through rotation and zooming interactions
	\item STL file export support for 3D printing
	\item Support for current browsers
	\item Deployment of a fully working version in a web server for the general public. Anybody with a Withings account should be able to use it an visualize their data.
\end{itemize}


The required functionality is beyond what it would be needed for a basic working prototype. The developed configurator in this work is a fully functional system that fulfills every requirement specified in this section. The stated requirements involved the implementation of other functionality to ensure the requirement was fulfilled. For example the login functionality implies the implementation of an Oauth authentication solution(explained later in section \ref{sub:ApiIntegration}) in order to integrate with the Withings API. Also because the configurator is dealing with personal data the security of the system is important, as a basic approach to address this, the configurator implements session handling.

In the following section the used technologies to tackle the extensive list of requirements will be presented.  

\subsection{Technology}
The web configurator was developed with a modern stack of technologies that enable the use of the JavaScript language not only for frontend but also for the backend, the database and graphics programming. The MEAN stack is a collection of technologies that enables developers to write web applications in a single programming language. MEAN is the acronym for the technologies  conforming the stack: \textit{MongoDB}\cite{mongodb}, \textit{Express}\cite{express}, \textit{AngularJS}\cite{angular} and \textit{Node.js}\cite{joyent2015node}. The acronym was first used by Valeri Karpov, MongoDB engineer, in a blog post\cite{meanstack} where he discussed the benefits of using the same language across all technologies involved. Among the benefits Karpov states their team have experienced an increment in the performance of the developed software and also in the team's productivity. In short the MEAN stack's frameworks will be introduced. 

\textbf{MongoDB} is an ``open-source, document database designed for ease of development and scaling''\cite{mongodb}. Document based are conformed of collections (the equivalent of tables) which can contain several documents (rows) that  to any schema are in essence JSON Objects (JavaScript Object Notation). MongoDB does not enforce any schema to the stored data, making it a very flexible environment to work with. In MongoDB queries are performed in JavaScript providing a seamless integration of the JavaScript language, from querying the information to structuring it in JSON objects. The stored JSON objects are encoded to the BSON format which stands for Binary JSON to enhance efficiency and provide additional data types. Even though working with a schema-less database brings great flexibility, it is recommendable to bring some consistency to the MongoDB documents. For this purpose an object modeling library called \textit{mongoose}\cite{mongoose} has found broad adoption in the developer community. Mongoose provides functionality to easily model the application's data with type casting, custom query functions and other niceties. To illustrate how a mongoose schema looks like, listing \ref{list:mongoose-schema} shows an excerpt of the user schema designed for the configurator.

\begin{lstlisting}[style=htmlcssjs, caption={An excerpt of the user mongoose schema},label=list:mongoose-schema]
// load the mongoose module
var mongoose = require('mongoose');
var Schema = mongoose.Schema;

var userSchema = new Schema({
  profile: {
    name: String, //data type definition
    gender: Number,
    age: Number,
    location: String
  },
// more fields... 
  settings: {
    startDate: Date,
    endDate: Date,
    show: Boolean
  },
  sculptures: {type: Array, "default": []}
});
\end{lstlisting} 

\textbf{Node.js}\cite{joyent2015node} and \textbf{Express}\cite{express} are the frameworks used to develop the web server of the application. Node.js is backend JavaScript, based on the ``V8'' chrome JavaScript runtime providing an event-driven architecture and non-blocking i/o allowing for high performance real time applications. \textit{Node.js} is conformed of basic modules that provide network functionality but it requires the developer to build everything from the ground up. To address this issue the JavaScript library \textit{Express} was developed. Express provides a rich set of features like routing, middleware support and HTTP utilities that allow rapid development of network applications. Listing \ref{list:node-server} shows how a simple server is implemented with Node.js and Express.

\begin{lstlisting}[style=htmlcssjs, caption={Basic Node.js and Express server example},label=list:node-server]
var express = require('express');
var app = express();

app.get('/', function (req, res) {
  res.send('Hello World!');
});

var server = app.listen(3000, function () {
  var host = server.address().address;
  var port = server.address().port;
  console.log('Example app listening at http://%s:%s', host, port);
});
\end{lstlisting}

\textbf{AngularJS}\cite{angular} is the fronted framework in the MEAN stack. \textit{AngularJS} is an opens-source project developed by Google for the development of Single Page Applications (SPAs) where the whole content of a web page is rendered in a single web page and the content is either completely loaded at the beginning or its injected according to user interactions. AngularJS is based on a Model-View-Whatever design pattern (MVW) that allows developers to choose the design pattern that better accommodates their coding style. Whichever design pattern is chosen, AngularJS enforces developers to structure code into modules providing a strong organization in the code base. The functionality provided by AngularJS ranges from bidirectional data-binding, routing, form validation, server communication tools and directives that allow the creation of custom HTML syntax. With frameworks like AngularJS SPA allow to take much of the server workload to the frontend. Another feature of AngularJS is its extensibility. For the purposes of this work the \textit{Angular Material}\cite{angularmaterial} was utilized to implement user interface controls and theming. The \textit{Angular Material} module is also developed by Google and it provides a solid framework for developing user interfaces utilizing the ``Material design'' philosophies. 

In the following sections the utilization the benefits of developing with the MEAN stack with other technologies will be highlighted through examples of modules developed for the web configurator. 

\subsection{Architecture}


\begin{figure}[h]
\captionsetup{width=\textwidth}
\begin{center}
  \includegraphics[width=\textwidth]{Configurator/img/Architecture}
  \caption{Activity Sculpture web configurator's architecture}
\label{fig:architecture}
\end{center}
\end{figure}

\begin{figure}[h]
\captionsetup{width=\textwidth}
\begin{center}
  \includegraphics[width=0.9\textwidth]{Configurator/img/flowdiagram_1}
  \caption{Activity Sculpture web configurator's flow diagram part 1}
\label{fig:flowdiagram}
\end{center}
\end{figure}

\begin{figure}[h]
\captionsetup{width=\textwidth}
\begin{center}
  \includegraphics[width=0.86\textwidth]{Configurator/img/flowdiagram}
  \caption{Activity Sculpture web configurator's flow diagram part 2}
\label{fig:flowdiagram}
\end{center}
\end{figure}

\subsection{Configurator}
\subsubsection{Sculpture Generation \& Rendering}
Real time visualization
\textbf{WebGL} \textit{Threejs}\cite{cabello2010three}

\subsubsection{Sculpture Manipulation}

\label{sub:sculpturegeneration}
\subsection{Backend}
\subsubsection{Withings API Integration}
\label{sub:ApiIntegration}

Authentication 
\textbf{Passport}\cite{passport} 
\textbf{OAuth} \cite{hammer2010oauth}

Querying user data
\textbf{Withings-API}\cite{withingsApi} 
\textit{RESTful Web services} \cite{Fielding:2000:PDM:337180.337228}
\textbf{WebRTC}\cite{webrtc}
\textbf{SocketIO}\cite{socketio} 



\subsubsection{Data Processing}
\subsection{Challenges}
\end{document}