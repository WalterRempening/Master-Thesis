%\iffalse\bibliography{ma-thesis.bib} \fi
\section{Introduction}
\label{ch:intro}
The presented work deals with two major topics: web customization platforms and
activity sculptures. For the former topic interaction processes and usability aspects
applied in current projects are of great interest as they provide a foundation on
which the author's prototype will be built upon. The latter will help the user
explore and engage with their activity data in a meaningful way, first in virtual and later in
physical 3D space. The following sections describe an in depth look at the
motivation of this work, the problems that are set to be resolved and the goals
to be achieved. To conclude this chapter a general overview of each chapter will be
provided. 

\subsection{Motivation}
In the process of modern data visualization one is repeatedly confronted with
the challenge of making sense of an abstract dataset by efficiently mapping data
variables onto a meaningful visual representation\cite{gee2005dynamic}
Advent of the Quantified self through technology advancements
Data Visualization and Activity Sculptures 
3D fabrication

history of web configurators
boom of mass customization of products
principal appliances
importance of web configurators in e commerce applications

Web technologies allow new interaction methods
Wrapping it up


\subsection{Goals}
The main goal of this work is to develop a system that can guides intuitively
the user in each process of the visualization of his activity data. This all
includes importing the activity data of the user and processing it in order to be
visualized in a sculpture which will be further manipulated to users preferences
and exporting it for 3D print. The aim of the web configurator is to perform all
these tasks providing the user the best experience possible. For this the development 
of interaction concepts, that guide users through each step of the configuration
process, plays an important role in the achievement of an enjoyable platform.
Another goal is to ensure the interaction concepts in the prototype are
understandable and easy to grasp. In order to achieve this goal users feedback
was taken into account through user studies and questionnaires. The diversity of
users chosen for the studies was made possible through local testers and through
an online demo of the prototype. Furthermore the design of an activity sculpture that shows
high variability in the configuration possibilities was an objective kept in mind
throughout the prototyping phase. In order for the system to respond fast to
user input a special set of technologies was needed. This work aims to take advantage of
current edge technologies by implementing them in the prototype. 


\subsection{Content overview}
The presented work takes the following structure. Chapter 2 presents current
configurators in different fields of the industry and academic research. Further
on current projects related to activity sculptures will be discussed. The final
section of the chapter presents an analysis of activity data sources
and current implementation of available fitness tracker APIs. In Chapter 3 the
prototype design process will be presented. For this sketches and concepts for
both sculptures and the configurator will be explained concluding with final
thoughts about the final decision making. Chapter 4 deals with the development
and implementation of the prototype. In this chapter the prototype's architecture and
special features will be discussed. Chapter 5 is focused on the design and
execution of the user study concluding with a discussion about the results and
findings of the study. Chapter 6 concludes this work and chapter 7 states the
ways on which this work can be further developed.
