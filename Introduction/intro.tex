%\iffalse\bibliography{ma-thesis.bib} \fi

\documentclass[Medieninformatik-arbeit.tex]{subfiles}
\begin{document}
\section{Introduction}
\label{ch:intro}
The presented work deals with two major topics: web customization platforms and
activity sculptures. For the former topic interaction processes and usability aspects
applied in current projects are of great interest as they provide a foundation on
which the author's prototype will be built upon. The latter will help the user
explore and engage with their activity data in a meaningful way, first in virtual and later in
physical 3D space. The following sections offer an background information to the topic
and a define the problem addressed in this work. To conclude this chapter a general
overview of each chapter will be provided. 

\subsection{Background \& Problem Definition}
At its core, the problem to solve in this work is a data visualization problem.
The field of data visualization is tightly interlaced with other fields such as
statistics, psychology, design, human-computer interaction, computer and cognitive
sciences to name a few. This makes it a science that requires a broad set of
skills to master. Due to its multidisciplinary nature it is also difficult to
define. A definition of data visualization suitable for this work would be the
one described by Card et al.\cite{Card:1999:RIV:300679}:``The use of
computer-supported, interactive, visual representations of abstract data to
amplify cognition''. In other words this definition states that the existent knowledge about a
specific dataset can be increased by mapping the data to a visual space and
by interacting with it through a computer system 

The interrogatives and curiosities about a dataset are the start point of
every visualization. This work aims to answer some questions about activity data and the
possible forms of representing it in a physical sculpture. But why activity
data? We are living in a time where there is more data flowing every second through the
internet than all the data stored in the internet 20 years
ago\cite{mayer2013big}. It is estimated that the measure of digital created data
will grow by a factor of 300 from 130 exabytes in 2005 to 40,000 exabytes in
2020\cite{gantz2012digital}. This forecast shows how the analysis of big data is 
playing an important role in the way how decisions are made in the industry. The
large scale nature of big data observing how millions of users behave can also
be put into a much smaller scale namely the quantified self. Big data and the
quantified self could be compared to a telescope and a microscope. Both use the
same principles to amplify the ability of humans to observe but on totally
different magnitudes. The quantified self is a movement of people that make use of
self-tracking tools to measure physical performance or vital signs for self
improvement and was first described by Wired editor Gary
Wolf\cite{wolf2009know}. Noticing the great momentum the movement was having
Gary Wolf founded together with his fellow editor Kevin Kelly the Quantified
Self Labs\cite{quantified:2015:Online}. Even though people have been
tracking themselves through the centuries and the improvement of performance by
solely knowing one is being observed has been also
been studied\cite{mccarney2007hawthorne}, the technological improvements in
sensor development has made the process of gathering activity data much easier.
So much that anybody with a smartphone can start tracking his own activity.
Advocates of the quantified self movement see in this practice a tremendous
potential for solving health challenges through big
data\cite{swan2013quantified}. 

It is therefore  of great interest to develop new
visual representations for the increasing amount of personal data available. The
challenge for these visualizations is that they can engage people in a more
deeper level as the data treated is a reflect of their own behavior. For this
purpose designers and engineers have been taking advantage of digital
fabrication systems, in particular 3D printing, to translate their
visualizations from the screen to physical space. Activity sculptures have shown
to be a suitable visualization for communicating abstract data into a tangible
object. While it can be discussed about the usefulness of such an object is
mainly determined by the information it can convey\cite{zhao2008embodiment}, 
activity sculptures have shown to influence people's behavior
positively\cite{khot2014understanding}. Other valuable characteristics of activity
sculptures is the interaction possibilities and their physical 
properties\cite{zhao2008embodiment}. Simply by being physically constructed they
can approach the natural instinct of interacting with objects through touch, 
by feeling its material and exploring its surface and structure.

As discussed before, one key aspect in the process of modern data visualization
is that the interaction with the visual representation of the data occurs on a
computer-aided system. One approach that has proved to be efficient in aiding
users to manipulate objects to their specific need is the configurator. This
tool supports the product configuration process satisfying every design and
configuration constrain set by the manufacturer\cite{hedin1998product}. 
The development of configuration systems originated from the mass customization
paradigm, that establishes a business model in which companies
massively manufacture individually customized 
goods\cite{felfernig2014knowledge}.
This paradigm is divided in different methods that offer each different degrees
of freedom to customers. Out of this method the
collaborative customization or the co-creation method uses of software
tools that allow the customer to transfer their preferences directly to the
product\cite{pine1999mass,piller2006user}. Most of the configurators are
deployed as web applications, in large due to the scalability and accessibility
of web systems. Advancements in web technologies allow manufacturers to build
more complex configuration systems. 
Because the configurator's degree of user friendliness can have a positive or
negative impact on the completion of a sale, it is important to companies to
understand how to guide the customer in a meaningful way to complete the
sale\cite{rolland2012commerce,abbasi2012s}. The work-flow proposed in research
is usable to ensuring the successful guidance of the user throughout the process
of developing a product suited to his needs. This same knowledge could be
applied to the development of a visualization system. 

To summarize this section, the motivation of this work could be resumed as
follows. The increasing desire to quantify every aspect of the self and the advancements in digital 
fabrication open the field for a new kind of visualizations residing 
in physical space out of the constrains of the screen. This is a challenge for
the data visualization field that can be addressed through co-creation production techniques and best
practices to develop a software tool that includes the user in
the process of achieving a visualization
that is aesthetically and functionally meaningful. 

\subsection{Goals}
The main goal of this work is to develop a system that can guides intuitively
the user in each process of the visualization of his activity data. This all
includes importing the activity data of the user and processing it in order to be
visualized in a sculpture which will be further manipulated to users preferences
and exporting it for 3D print. The aim of the web configurator is to perform all
these tasks providing the user the best experience possible. For this the development 
of interaction concepts, that guide users through each step of the configuration
process, plays an important role in the achievement of an enjoyable platform.
Another goal is to ensure the interaction concepts in the prototype are
understandable and easy to grasp. In order to achieve this goal users feedback
was taken into account through user studies and questionnaires. The diversity of
users chosen for the studies was made possible through local testers and through
an online demo of the prototype. Furthermore the design of an activity sculpture that shows
high variability in the configuration possibilities was an objective kept in mind
throughout the prototyping phase. In order for the system to respond fast to
user input a special set of technologies was needed. This work aims to take advantage of
current edge technologies by implementing them in the prototype. 


\subsection{Content overview}
The presented work takes the following structure. Chapter 2 presents current
configurators in different fields of the industry and academic research. Further
on current projects related to activity sculptures will be discussed. The final
section of the chapter presents an analysis of activity data sources
and current implementation of available fitness tracker APIs. In Chapter 3 the
prototype design process will be presented. For this sketches and concepts for
both sculptures and the configurator will be explained concluding with final
thoughts about the final decision making. Chapter 4 deals with the development
and implementation of the prototype. In this chapter the prototype's architecture and
special features will be discussed. Chapter 5 is focused on the design and
execution of the user study concluding with a discussion about the results and
findings of the study. Chapter 6 concludes this work and chapter 7 states the
ways on which this work can be further developed.
\end{document}
