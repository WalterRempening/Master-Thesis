 % -*- root: ../medieninformatik-arbeit.tex -*-
\documentclass[../medieninformatik-arbeit.tex]{subfiles}
\begin{document}
\section{Future Work}
\label{ch:future}
The web configurator can be extended for future work by implementing support for more device like tablets and smartphones. Due to time constraints it was not possible to add responsive functionality and the actual configurator is not adapted to mobile devices. In order to keep providing an enjoyable experience while using the configurator, new interaction concept should be developed for mobile devices as screen real state is scarce and maybe some built in sensors could be utilized to operate UI controls. 

In order to overcome the limitations of the reduced user base of Withings devices would be the support of more fitness trackers. In this way more users can have access to the configurator. Due to the modular architecture of the web configurator supporting more fitness trackers could be possible with not much difficulty. It would only require to write a formatting module for a handling the data format for a specific vendor. It would be also interesting to visualize other features of the API like goals or notifications. In the Withings API it was possible to get more detailed activity data during the day, but for this a special request had to be send. Due to the uncertainty of the duration of the process it was discarded. Also by utilizing more advanced fitness trackers with GPS support activity sculptures like like the activity landscape could be implemented. The login functionality could also be improved by making the session handling implementation more solid as for now no logout functionality is implemented. 

Also because of time constraints it was not possible to implement a presentable welcome page as designed in the prototypes. This feature would help to provide more information about the project to users. The welcome page could contain the tutorial and gallery sections for showcasing what the configurator is possible to do. Another way of expanding the web configurator would be the integration with social media. When users finish configuring a sculpture they could also share it on social media and send a link to friends or followers. This could provide a fresh way of visualizing data after a workout instead of only showing some standard charts or maps. 

In the dashboard area the configurator could be improved by extending the gallery section. Until now only the name, date and color of the sculpture are visualized in a grid. It would be helpful to users revisiting their sculptures to have a preview window showing up, where a summary of the activity data is provided and also a visualization of the sculpture that can be navigated. 

The configurator view could be further expanded by adding a chart of the selected variables beneath the sculpture to provide a reference point to users about the selected data. The date range slider could be expanded to filter dates from the beginning too, as for no it only filters the newest date. Being able to filter the desired date range could provide users with the freedom of choosing a specif time span for the data. Furthermore the configurator could be expanded to offer users more variety of sculptures. This could be implemented by expanding the \textit{wcModel} service to manage the generation of multiple sculptures. This Another addition to the configurator would be a further step where users can chose from different accessories for expanding the sculpture to daily use objects like rings, mugs, earrings or other artifacts. 

The web configurator of this work focused on activity data generated by fitness trackers that are more focused toward certain kinds of activity like jogging or normal activity like walking and climbing stairs. It would interesting to explore the design of activity sculptures for specific sports to which would make the sculpture more configurator more appealing to users looking to visualize their activity.

To conclude, the web configurator developed in this work offers a good starting point for further improvements as it already has the required functionality to visualize activity sculptures from data gathered from other web services. Developing new design concepts for activity sculptures is still a field with lot of space for exploration.

\end{document}