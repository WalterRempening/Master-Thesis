\chapter{Background & Relatedwork}
\label{ch:related}
\section{Related Work}


Siehe was cooles~\ref{fig:beispiebild} oder einschlägige Literatur, z.B.
\cite{ahu61} oder \cite{ab94} oh ja ziemlich geil das ding
\bigskip % Größerer Abstand zum vorherigen Absatz


\subsection{Medien}

\begin{figure}
  \begin{center}\LARGE [BILD]\end{center}
  \caption{Noch ein Bild}
  \label{fig:beispielbild2}
\end{figure}

\begin{itemize}
  \item Gesellschaftliche Medien
  \item Technische Medien
\end{itemize}


\subsection{Informatik}


\subsection{Medieninformatik}

\begin{description}
  \item[Medienwirkung:] Ein Spezialfach der Kommunikationswissenschaft. Für das erfolgreiche Studium des Anwendungsfachs Mediengestaltung ist eine künstlerische Begabung erforderlich.
  \item[Medienwirtschaft:] Ein Spezialfach der Betriebswirtschaftslehre
  \item[Mediengestaltung:] Ein Spezialfach der Kunstwissenschaft
\end{description}

\subsubsection{Was Sie schon immer wissen wollten, aber nie zu fragen
  wagten}

\paragraph{Überschrift}
Diese Überschrift erscheint fettgedruckt am Anfang des Absatzes.

\subsubsection{Was Sie nicht wissen wollten}

Text text textextext\footnote{Oder so ähnlich}.
What ever you do

